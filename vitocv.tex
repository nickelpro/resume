%%%%%%%%%%%%%%%%%%%%%%%%%%%%%%%%%%%%%%%%%
% Developer CV
% LaTeX Class
% Version 2.0 (12/10/23)
%
% This class originates from:
% http://www.LaTeXTemplates.com
%
% Authors:
% Omar Roldan
% Based on a template by  Jan Vorisek (jan@vorisek.me)
% Based on a template by Jan Küster (info@jankuester.com)
% Modified for LaTeX Templates by Vel (vel@LaTeXTemplates.com)
%
% License:
% The MIT License (see included LICENSE file)
%
%%%%%%%%%%%%%%%%%%%%%%%%%%%%%%%%%%%%%%%%%

%----------------------------------------------------------------------------------------
%	PACKAGES AND OTHER DOCUMENT CONFIGURATIONS
%----------------------------------------------------------------------------------------

\documentclass[9pt]{developercv} % Default font size, values from 8-12pt are recommended
\usepackage{multicol}
\setlength{\columnsep}{0mm}
%----------------------------------------------------------------------------------------
\usepackage{lipsum}


\begin{document}

%----------------------------------------------------------------------------------------
%	TITLE AND CONTACT INFORMATION
%----------------------------------------------------------------------------------------

\begin{minipage}[t]{0.5\textwidth}
	\vspace{-\baselineskip} % Required for vertically aligning minipages

	{ \fontsize{16}{20} \textcolor{black}{\textbf{\MakeUppercase{Vito Gamberini}}}} % First name

	\vspace{6pt}

	{\Large Developer $\sim$ Computer Engineer} % Career or current job title
\end{minipage}
\hfill
\begin{minipage}[t]{0.2\textwidth} % 20% of the page width for the first row of icons
	\vspace{-\baselineskip} % Required for vertically aligning minipages

	% The first parameter is the FontAwesome icon name, the second is the box size and the third is the text
	\icon{Globe}{11}{\href{https://vito.nyc}{vito.nyc}}\\
	\icon{Phone}{11}{631-912-5918}\\
	\icon{MapMarker}{11}{New York City, NY}\\

\end{minipage}
\begin{minipage}[t]{0.27\textwidth} % 27% of the page width for the second row of icons
	\vspace{-\baselineskip} % Required for vertically aligning minipages

	\icon{Envelope}{11}{\href{mailto:vito@gamberini.email}{vito@gamberini.email}}\\
	\icon{Github}{11}{\href{https://github.com/nickelpro}{/nickelpro}}\\
	\icon{LinkedinSquare}{11}{\href{https://www.linkedin.com/in/vito-gamberini}{/in/vito-gamberini}}\\

\end{minipage}


%----------------------------------------------------------------------------------------
%	INTRODUCTION, SKILLS AND TECHNOLOGIES
%----------------------------------------------------------------------------------------

\begin{minipage}[t]{0.46\textwidth}
	\cvsect{Summary}
	\vspace{-6pt}

	Passionate about cross-domain digital systems engineering and performance
	applications. Formerly a submarine reactor operator with the U.S. Navy,
	experienced with high-speed, high-stakes engineering teams.
\end{minipage}
\hfill % Whitespace between
\begin{minipage}[t]{0.465\textwidth}
	\cvsect{Skills}
	\vspace{-6pt}

	\begin{minipage}[t]{0.2\textwidth}
		\textbf{Languages:}
	\end{minipage}
	\hfill
	\begin{minipage}[t]{0.73\textwidth}
		C++, C, Python, Javascript, Bash, SQL, SystemVerilog, x86 Assembly
	\end{minipage}
	\vspace{4mm}

	\begin{minipage}[t]{0.2\textwidth}
		\textbf{Technologies:}
	\end{minipage}
	\hfill
	\begin{minipage}[t]{0.73\textwidth}
		CMake, git, vcpkg, LLVM, Linux, Docker, GH Actions, Codecov, Verilator, LaTeX
	\end{minipage}

\end{minipage}

%----------------------------------------------------------------------------------------
%	Projects
%----------------------------------------------------------------------------------------
\cvsect{Recent Projects}
\begin{fullentrylist}
	\fullentry
	{nanoroute (C++ / Python)}
	{
		\icon{Globe}{}
		{\href{https://nanoroute.dev/}{\textbf{Website}}}
	}
	{
		The fastest HTTP URL router available for Python, outpaces industry-standard
		solutions such as Werkzeug and Starlette by a factor of \textbf{20x} to
		\textbf{30x}. Supports typical use-cases such as PEP 3333 WSGI services.
	}
	\fullentry
	{Velocem (C++ / Python)}
	{
		\icon{Github}{}
		{\href{https://github.com/nickelpro/velocem}{\textbf{Github Link}}}
	}
	{
		Current research project under active development, a hyperspeed Python web
		development framework. Benchmarks as the lowest latency Python application
		server implementation publicly available.
	}
	\fullentry
	{E20 Binutils Suite (C++)}
	{\textbf{Proprietary}}
	{
		Complete binutils suite for the E20 pedagogical machine language, including
		assembler, disassembler, linker, simulator, and debugger. Used by hundreds of
		students in the New York University Computer Architecture course each
		semester.
	}
	\fullentry
	{NYU Processor Design Build System (CMake / C++)}
	{
		\icon{Github}{}
		{\href{https://github.com/NYU-Processor-Design/component-template}{\textbf{Github Link}}}
	}
	{
		System for building, testing, generating coverage reports, and packaging
		SystemVerilog components. Adapts vcpkg and CMake to support a different
		language paradigm while seamlessly integrating with the Verilator C++ code
		generation tool.
	}
	% \fullentry
	% {PurdNyUart (SystemVerilog)}
	% {
	% 	\icon{Github}{}
	% 	{\href{https://github.com/NYU-Processor-Design/PurdNyUart}{\textbf{Github Link}}}
	% }
	% {
	% 	Complete universal asynchronous receiver/transmitter design used as the
	% 	initial bootstrapping interface for Purdue's AFTx07 chip. Features a novel
	% 	digital baud rate generator, as well as 100\% automated test coverage of
	% 	the entire design.
	% }
\end{fullentrylist}

%----------------------------------------------------------------------------------------
%	EXPERIENCE
%----------------------------------------------------------------------------------------
\vspace{-10 pt}
\cvsect{Experience}
\begin{entrylist}
	\entry
	{6/2024 -- Present}
	{Research \& Development Engineer}
	{\textbf{Kitware, Inc.}}
	{\vspace{-10pt}
		\begin{itemize}[noitemsep,topsep=0pt,parsep=0pt,partopsep=0pt, leftmargin=-1pt]
			\item Developed multiple upstream CMake features, including native
						pkg-config support, automatic dependency exports, and improvements
						to the argument parser
			\item Ported dozens of production codebases to modern CMake \& CI usage.
			\item Trained teams on modern CMake usages and build practices.
		\end{itemize}
	}
	\entry
	{9/2021 -- 5/2024}
	{Teaching Assistant}
	{\textbf{NYU Tandon School of Engineering}}
	{\vspace{-10pt}
		\begin{itemize}[noitemsep,topsep=0pt,parsep=0pt,partopsep=0pt, leftmargin=-1pt]
			\item Developed and presented lecture materials on topics including
			      intermediate-level C/C++ usage, processor instruction decoding,
			      pipelining, speculative execution, and cache coherence protocols.
		\end{itemize}
	}
	\entry
	{3/2014 -- 6/2020}
	{Submarine Nuclear Reactor Operator}
	{\textbf{United States Navy}}
	{\vspace{-10pt}
		\begin{itemize}[noitemsep,topsep=0pt,parsep=0pt,partopsep=0pt, leftmargin=-1pt]
			\item Operated and maintained I\&C systems directly involved with the
			      primary reactor plant
			\item Maintenance Lead for the upgrade of primary reactor equipment.
			      Developed procedures for unique maintenance operations and
						supervised hundreds of routine maintenance tasks.

		\end{itemize}
	}
\end{entrylist}

%----------------------------------------------------------------------------------------
%	EDUCATION
%----------------------------------------------------------------------------------------
\vspace{-10 pt}
\cvsect{Education}
\begin{entrylist}
	\entry
	{9/2020 - 5/2024}
	{Computer Engineering, BSc}
	{\textbf{NYU Tandon School of Engineering}}
	{
		\textbf{Relevant Coursework:} Operating Systems,
		Computer Architecture, Embedded Programming,
		Digital Logic \& State Machine Design,
		Programming Languages \& Implementation, Unix System Programming
	}
\end{entrylist}

%----------------------------------------------------------------------------------------
%	Writings
%----------------------------------------------------------------------------------------
\vspace{-10 pt}
\cvsect{Select Writings}
\begin{entrylist}
	\entry
	{Jun 2024}
	{Make it Async: Building Shared Async Resources with ASIO}
	{
		\icon{Globe}{}
		{\href{https://blog.vito.nyc/posts/make-it-async/}{\textbf{Article Link}}}
	}
	{
		A demonstration-by-example of managing shared resources in async C++
		environments, leveraging C++20 coroutines and \texttt{boost::asio} strand
		executors.
	}
	\entry
	{Dec 2023}
	{Balm in GILead: Fast string construction for CPython extensions}
	{
		\icon{Globe}{}
		{\href{https://blog.vito.nyc/posts/gil-balm}{\textbf{Article Link}}}
	}
	{
		An unorthodox approach to optimizing Python C extensions that operate on
		Python strings. The described technique achieves a \textbf{5x} improvement
		on single-threaded benchmarks and up to a \textbf{20x} improvement on
		multi-threaded benchmarks.
	}
	\entry
	{Dec 2022}
	{Modern CMake Packaging: A Guide}
	{
		\icon{Globe}{}
		{\href{https://blog.vito.nyc/posts/cmake-pkg}{\textbf{Article Link}}}
	}
	{
		Complete guide to packaging facilities of modern CMake with a focus on
		correctness. Places an emphasis on understanding the mechanisms of
		packaging and discoverability instead of rote copy-pasting of build system
		boilerplate.

	}
	% \entry
	% {Jul 2022}
	% {Upside Down Polymorphic Inheritance: Leveraging P2162 for Fun \& Profit}
	% {
	% 	\icon{Globe}{}
	% 	{\href{https://blog.vito.nyc/posts/p2162}{\textbf{Article Link}}}
	% }
	% {
	% 	An introduction to the usage and applications of the C++20 feature that
	% 	allows for inheriting from and extending \texttt{std::variant}. This enables value-semantics to be used in conjunction with closed-set polymorphic types.
	% }
\end{entrylist}

%----------------------------------------------------------------------------------------

\end{document}
